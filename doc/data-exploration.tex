\section{Data exploration and visualisation}
\label{secn:data-exploration}

Here we will examine the dataset in more detail to understand the patterns, distributions, and relationships. In this exercise, we will use more of the visual tools available through several Python libraries to identify potential biases, explore correlations between variables, and uncover insights that may influence the outcomes of predictive models. 

\subsection {Demographic analysis}

We begin this analysis by segmenting the dataset based by race and gender.





\begin{figure}[H]
	\centering
	% First image
	\begin{subfigure}[b]{0.45\linewidth}
		\centering
		\includegraphics[width=\linewidth]{img/race_percentages_pie.png}
	\end{subfigure}
	\hfill
	% Second image
	\begin{subfigure}[b]{0.45\linewidth}
		\centering
		\includegraphics[width=\linewidth]{img/sex_percentages_pie.png}
	\end{subfigure}
	\caption{Race and Sex breakdown for the dataset.}
	\label{fig:race-sex-breakdown}
\end{figure}


%% First image
%\begin{figure}
%	\centering
%	\includegraphics[width=0.7\linewidth]{img/race_percentages_pie.png}
%	\caption{Caption for the first image}
%	\label{fig:image1}
%\end{figure}
%\hfill
%% Second image
%\begin{figure}
%	\centering
%	\includegraphics[width=0.7\linewidth]{img/sex_percentages_pie.png}
%	\caption{Caption for the second image}
%	\label{fig:image2}
%\end{figure}


By examining the racial composition of the dataset, we observe the following:

\begin{itemize}
	\item Over half of the test dataset is composed of African-American individuals, suggesting that the dataset may be imbalanced, with a disproportionate representation of one racial group.
	\item Asians and Native Americans each makeup only 0.2\% of the dataset; this underrepresentation might raise some concerns as it may lead to challenges in statistical analysis or machine learning models. Such concerns include the lack of reliability or significance for these groups due to insufficient data.
\end{itemize}

Figure \ref{fig:race-sex-breakdown} also shows our dataset's male/female split, with females comprising only 19.8\%. It is, therefore, evident that the female group is underrepresented, which can lead to biased models as models may overfit male patterns and underperform on females and misleading conclusions as insights derived might generalise poorly for the female subgroup.


\subsection{Age distribution analysis}


We used a boxplot to illustrate the age patterns across racial groups, helping to identify central values, spread, and any anomalies.

\begin{figure}[H]
	\centering
	\includegraphics[width=0.9	\linewidth]{img/age_by_race_boxplot}
	\caption{}
	\label{fig:age-by-race-boxplot}
\end{figure}



\begin{table}[H]
	\centering
	\begin{tabular}{|l|l|p{4cm}|}
		\hline
		Race & Median Age & Distribution Description \\ \hline \hline
		African-American & $\approx$30 years & Concentrated in the 20–40 range, with a relatively narrow spread. We notice outliers above 60 years,  indicating fewer older individuals. \\ \hline
		Caucasian & $\approx$40 years & Broader age range, from 20 to 70+ years. We notice more older individuals (upper outliers), making this group appear older on average. \\ \hline
		Hispanic & $\approx$30–35 years & Moderately broad spread, with most individuals between 20 and 50 years. \\ \hline
		Asian & $\approx$30 years & Narrow distribution, concentrated between 25 and 40 years. No outliers. \\ \hline
		Native American & $\approx$33 years & Very tight distribution, with all ages clustered closely around the median (little variability). It is important to note that this group accounts for a tiny portion of the population. \\ \hline
		Other & $\approx$35–40 years & Similar to Caucasians but with slightly fewer older individuals. The IQR shows a widespread. \\ \hline
	\end{tabular}
\end{table}



\subsection{Analysing distributions}


Next, we plotted the distributions of all the features in our dataset; this is depicted in Figure \ref{fig:distribution-plots} . From these histograms, we notice 
\begin{itemize}
	\item The age distribution shows a right-skewed pattern, with most individuals falling in the younger age ranges (20–40 years).
	
	\item There is a significant over-representation of certain racial groups, particularly African Americans, which could indicate potential biases in the dataset's sampling.
	
	\item Most individuals have zero juvenile felony counts, zero juvenile misdemeanour counts, and no recorded "other" juvenile offences, with each distribution rapidly declining for higher counts.
	
	\item A large proportion of individuals have a low number of prior offences, but there is a long tail indicating some individuals have a significant number of priors.
	
	\item Most individuals have relatively short jail durations, with a few experiencing significantly longer durations.
	
	\item The distribution of days between screening and arrest is clustered around zero, with few extreme outliers on both ends.
	
	\item The decile scores appear relatively evenly distributed, but slight patterns suggest clustering at specific score levels (e.g., lower decile scores are slightly more frequent).
	
	\item Two-year recidivism plot shows a near-equal distribution, indicating a balanced dataset for recidivism outcomes.
\end{itemize}


\begin{figure}[H]
	\centering
	\includegraphics[width=0.9\linewidth]{img/distribution-plots}
	\caption{}
	\label{fig:distribution-plots}
\end{figure}


\subsection{Correlation analysis}

The correlation matrix heatmap shown in Figure \ref{fig:correlationmatrixheatmap} was created to gather more insights and identify the areas of high correlation. This plot raises a number of interesting observations, namely:

\begin{figure}[H]
	\centering
	\includegraphics[width=0.7\linewidth]{img/correlation_matrix_heatmap}
	\caption{Correlation matrix heatmap.}
	\label{fig:correlationmatrixheatmap}
\end{figure}


\begin{itemize}
	\item Individuals with more prior offences tend to have higher risk scores, hence prior criminal behaviour is a key factor in risk assessment models. The number of prior offences also correlates positively with recidivism so it is true; individuals with more prior offences tend to re-offend more often.
	
	\item Older individuals tend to have lower risk scores, that is age may be inversely related to the risk of recidivism, with younger individuals being assessed as higher risk. In addition, older individuals are also less likely to recidivate, supporting this general trend.
	
	\item Individuals with higher risk scores are often recidivate within two years, suggesting that the risk score (\textbf{\texttt{decile\_score}}) is a predictor, to a certain extent, of recidivism.
	
	\item A correlation of 0.30 indicates that being African-American is moderately associated with higher decile scores, raising potential concerns about racial bias in the scoring system. At the same time, Caucasian individuals correlate -0.19 and, therefore, are less likely to receive higher risk scores.
	
	\item The time spent in jail has only a small positive relationship with the likelihood of re-offending within two years
	
	\item  Juvenile felony, misdemeanour, and other counts are positively correlated, indicating that individuals with one type of juvenile record will often have other types.
\end{itemize}

Figure \ref{fig:pairplot} is a pair plot created to substantiate these observations further to show the relationships among the selected numerical features, with the recidivism outcome (\textbf{\texttt{two\_year\_recid}}) as the hue.

\begin{figure}
	\centering
	\includegraphics[width=0.7\linewidth]{img/pair_plot}
	\caption{Pair plot for important features}
	\label{fig:pairplot}
\end{figure}


\begin{itemize}
	\item \textbf{\texttt{age}} vs \textbf{\texttt{priors\_count}}. Younger individuals tend to have fewer prior offences, but the prior count is scattered as age increases, showing that younger offenders tend to continue having problems with the judicial system.
	
	\item \textbf{\texttt{age}} vs \textbf{\texttt{decile\_score}}. Older individuals tend to have lower decile scores. Younger individuals are associated with higher scores.
	
	\item \textbf{\texttt{priors\_count}} vs \textbf{\texttt{decile\_score}}. Positive trend: Higher priors count leads to higher decile scores, suggesting a strong correlation, indicating the tendency for offenders to be viewed as a risk community    
	
	\item The distribution of \textbf{\texttt{decile\_score}}. Two-year recidivists tend to cluster at higher decile scores (6–10 range). Non-recidivists are spread more evenly across scores.        
\end{itemize}



\subsection{Analysis of decile score relationship with race}

The COMPAS dataset has been extensively discussed in recent years regarding the fairness of the scores assigned to individuals, especially considering the race component. Although the tool does not use race as one of the features in decile score prediction, there are concerns that race can be associated indirectly with other features. In this section, we will look at the data from the race point of view to see what patterns we can deduce from the dataset.

As a first step, we create a boxplot to compare the distribution of \textbf{\texttt{decile\_score}} across different racial groups. 

\begin{figure}[H]
	\centering
	\includegraphics[width=0.7\linewidth]{img/decile_score_by_race_boxplot}
	\caption{Risk score distribution by race.}
	\label{fig:decilescorebyraceboxplot}
\end{figure}


From this plot, we notice that African Americans have higher median scores and broader distributions, which suggests a potential bias in the COMPAS scoring system. In addition, Hispanics, Asians, and Other groups scored lower on average, which may indicate differences in the risk assessment process or underlying data inputs.


We then investigated the correlation of \textbf{\texttt{decile\_score}} with \textbf{\texttt{two\_year\_recid}} by race:

\begin{figure}[H]
	\centering
	\includegraphics[width=0.9\linewidth]{img/correlation_by_race}
	\caption{Correlation of \textbf{\texttt{decile\_score}} with \textbf{\texttt{two\_year\_recid}} by race}
	\label{fig:correlationbyrace}
\end{figure}

\begin{itemize}
	\item Asians and Native Americans show the highest correlations, indicating that, for these groups, COMPAS scores align more closely with observed recidivism outcomes. However, it is important to remember that these groups comprise a tiny percentage of the population.
	
	\item African-Americans, Caucasians, and Others have moderate correlations, so COMPAS scores are only somewhat predictive for these groups but not as strongly as for Asians or Native Americans.
	
	\item The correlation between \textbf{\texttt{two\_year\_recid}} and \textbf{\texttt{decile\_score}} for Hispanics is 0.22, the lowest among the groups, suggesting that for Hispanic individuals, the COMPAS scores are less predictive of actual recidivism outcomes than other racial groups. The low correlation for the Hispanic group is an area of concern because if the COMPAS scores do not accurately predict recidivism for this ethnicity, this could mean that the scoring system may overestimate or underestimate their actual risk, which could lead to misclassification of individuals, leading to potential unfair treatment, for example, harsher parole conditions or sentencing, or allowing more criminals on the streets.
\end{itemize}

The distribution of \textbf{\texttt{decile\_score}} for Hispanic individuals was then compared with that of other racial groups using kernel density plots (KDE). 

\begin{figure}[H]
	\centering
	\includegraphics[width=0.7\linewidth]{img/decile_score_by_race_hispanic}
	\caption{Risk score distribution for Hispanics compared with the risk score distribution of the dataset.}
	\label{fig:decilescorebyracehispanic}
\end{figure}


The plots show that the peak density for Hispanic individuals occurs at a lower score of around 2. In contrast, non-Hispanic individuals have a flatter distribution extending to higher scores, albeit peaking in the same region, suggesting that Hispanic individuals are more likely to receive lower COMPAS scores than other individuals.

One has to note that we have previously seen that Hispanic individuals account for 8.8\% of the population, which is significantly smaller compared to African-American and Caucasian groups.
Although by any means not conclusive, these aspects raise questions about how risk scores are calibrated for underrepresented groups.

\subsubsection{Analysis of the African-American sector}


We created another density plot to compare the distribution of risk scores between African-American individuals and non-African-American individuals.

\begin{figure}[]H
	\centering
	\includegraphics[width=0.7\linewidth]{img/decile_score_by_race_african_american}
	\caption{Risk score distribution for African-American compared with the risk score distribution of the dataset.}
	\label{fig:decilescorebyraceafricanamerican}
\end{figure}


Here, we notice that African Americans are more likely to receive risk scores in the higher decile range, exceeding a value of six, than non-African Americans, suggesting that the COMPAS tool systematically assigns higher risk scores to African Americans. As noticed previously, non-African-American individuals show a strong peak at around a score of 2, with fewer individuals in this group receiving scores in the higher ranges.

This disparity in score distributions raises concerns about potential bias in the COMPAS scoring system. African Americans appear to be disproportionately classified as higher risk, a factor which could impact downstream decisions such as sentencing or parole.

\subsection{Robustness of \textbf{\texttt{decile\_score}}}

As a final analysis, we compare \textbf{\texttt{decile\_score}} with \textbf{\texttt{two\_year\_recid}} to see how many high-risk individuals receded and how many individuals categorised as low-risk did not. This metric is very powerful, as it assesses the validity of the predicted risk metric. Throughout this study, we will also use this test to validate our predicted scores for the machine learning models.

An important decision in this exercise is the selection of the decile score threshold that will dictate that values above it are more likely to recidivate and those below it not. In order to do this, we compared the decile score and two-year record for multiple thresholds. We examined the number of true negatives (the individuals with a decile score lower than the threshold and did not recidivate), true positives (those with a decile score higher or equal than the threshold and did recidivate) and false positives and negatives. We then used these results to calculate the sensitivity, specificity, precision and accuracy:

\begin{table}[h!]
	\centering
	\begin{tabular}{|c|c|c|c|c|}
		\hline
		\textbf{Threshold} & \textbf{Sensitivity} & \textbf{Specificity} & \textbf{Precision} & \textbf{Accuracy} \\ \hline
		4                               & 0.628659             & 0.670551             & 0.60941            & 0.651707          \\ \hline
		5                               & 0.521186             & 0.766299             & 0.645823           & 0.656039          \\ \hline
		6                               & 0.411017             & 0.83811              & 0.674889           & 0.645989          \\ \hline
		7                               & 0.3047               & 0.897323             & 0.708147           & 0.63074           \\ \hline
		8                               & 0.196071             & 0.938268             & 0.721986           & 0.604401          \\ \hline
	\end{tabular}
	\caption{Performance Metrics for Different Decile Score Thresholds}
	\label{tab:performance_metrics}
\end{table}

Following this procedure, a threshold of 5 was decided to provide the best balance of these metrics. With this threshold, the total accuracy and precision are moderate, indicating that while the model performs reasonably well overall, there is room for improvement. We notice a low sensitivity compared to specificity; the model is better at identifying non-recidivists than recidivists.


The metrics for each racial group were then calculated with this threshold; 

\begin{itemize}
	\item The Caucasian group has a relatively high specificity but low sensitivity, indicating the COMPAS model more effectively avoids false positives but struggles to identify true positives. 

	\item The African-American group shows higher sensitivity but lower specificity compared to Caucasians. This suggests the model identifies more recidivists among African Americans but at the cost of more false positives.


	\item The Hispanic group has low sensitivity, meaning the model struggles significantly to identify true positives in this group. Precision is also the lowest, indicating a high false-positive rate.
\end{itemize}

The significant disparity in sensitivity and specificity across racial groups highlights potential fairness issues in the model's predictions. For example, the model disproportionately favours Caucasians and Asians in terms of specificity while penalizing African-Americans with higher false-positive rates.




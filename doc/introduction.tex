\section{Introduction}
	This exercise explores the well-known COMPAS dataset using several machine learning techniques, where we will also examine the ethical implications of predictive risk assessment models. In this study, we will apply techniques discussed in ICS5110, such as imputation and encoding, to assist in data preparation. Additionally, we trained a neural network, logistic regressor, and k-nearest neighbours model on the dataset to predict whether a person is likely to recidivate within two years.
	
	The selected dataset is particularly interesting because, in addition to the \textbf{\texttt{two\_year\_recid}} label—which indicates whether an individual has reoffended within two years of an initial charge, it also includes the COMPAS \textbf{\texttt{decile\_score}}, a risk rating predicting the likelihood of recidivism. This allows us to compare our findings with the COMPAS system and analyze the nuances of applying different techniques.



\subsection{History of the COMPAS tool}

	The COMPAS (Correctional Offender Management Profiling for Alternative Sanctions)\cite{wikipediaCOMPAS} dataset and tool have a controversial history rooted in its use for assessing the likelihood of recidivism among criminal defendants. Developed by Northpointe, COMPAS gained widespread adoption in the U.S. judicial system for pretrial risk assessments and sentencing decisions. This tool is helpful in various stages of the criminal justice process, including bail, sentencing, and parole decisions.
	
	However, in 2016, an investigative report by ProPublica\cite{larson2016compas} revealed significant racial biases in the tool's predictions. The report found that COMPAS disproportionately labelled Black defendants as high risk for reoffending while underestimating the risk for white defendants, even when both groups had similar criminal histories. This revelation sparked a broader debate about using algorithmic tools in criminal justice and their transparency and fairness. 
	
	The COMPAS tool has not been directly the subject of lawsuits, but its use in judicial decisions has led to legal challenges. For instance, in State v. Loomis (2016)\cite{harvard2017loomis}, the Wisconsin Supreme Court upheld using COMPAS in sentencing. However, judges must be informed about its limitations, particularly its proprietary nature and potential biases. The case highlighted the broader tension between the utility of predictive algorithms and their application's need for accountability and fairness.

\subsection{The COMPAS dataset}

The dataset that originates from the COMPAS tool is widely used in criminology and machine learning studies. 

The dataset contains attributes such as demographic information, prior charges, juvenile records, and risk scores, including the widely analysed decile score, which categorises individuals into ten different risk groups. 

The decile score is a critical feature, assigning a numerical value to an individual's likelihood of reoffending. Other important features include the number of prior offences \textbf{\texttt{priors\_count}} and the type of offence \textbf{\texttt{c\_charge\_degree}}, provide context for these predictions. At the same time, the label \textbf{\texttt{two\_year\_recid}} indicates whether an individual reoffended within two years of their COMPAS assessment.

While the dataset has been instrumental in research aimed at understanding and improving risk prediction models, it has also been the subject of extensive scrutiny due to its implications for fairness and equity in the justice system. A couple of thoughts resulting from this scrutiny include:


\begin{itemize}
	\item Multiple studies, including the influential ProPublica investigation in 2016\cite{larson2016compas}, have highlighted racial disparities in the COMPAS predictions. African-American defendants were found to be nearly twice as likely as Caucasian defendants to be labelled as high-risk for recidivism but not reoffend. Conversely, Caucasian defendants were more likely to be classified as low-risk but later reoffend, raising concerns about systemic bias embedded in the algorithm, which could exacerbate existing inequalities in the justice system.
	
	\item The COMPAS tool operates as a proprietary black-box model, meaning its internal workings and feature weights are not disclosed to the public or even to the defendants it evaluates. This lack of transparency prevents meaningful scrutiny and accountability, leaving users unable to fully understand or challenge the tool's predictions.
	
	\item The COMPAS algorithm relies on historical criminal justice data, which may reflect social and systemic biases. For example, law enforcement practices that can result in sentencing disparities can all influence the patterns observed in the data. Using such data as input, the COMPAS tool risks perpetuating these biases into an electronic tool.
	
	\item Some features in the COMPAS dataset, such as age and criminal history, are static and cannot change over time, as this data is based on the date of the COMPAS assessment. We can argue that these features in risk predictions without considering the period after the COMPAS assessment undermines the potential for individuals to reform and leads to insensible punitive outcomes.
	
	\item The ethical implications of using predictive algorithms in high-stakes decisions, such as sentencing and parole, constitute a significant area of concern. The potential for false positives can lead to unjustly harsher treatment, while false negatives can impact public safety.
	
	\item The dataset available for research purposes is a reduced version of the original COMPAS data, with several features anonymised or removed. Missing important data introduces limitations for academic studies aiming to replicate or validate the findings from real-world COMPAS applications.
\end{itemize}


The criticism of the COMPAS tool emphasises the challenges of deploying machine learning systems in sensitive domains like justice. These challenges are not unique to COMPAS but highlight broader issues in applying algorithmic decision-making tools in socially important contexts. They highlight the need for transparency, fairness-aware modelling techniques, and careful ethical evaluations when designing and implementing such tools.

\subsection{Objectives of this work}

The objective of this project is to assess the accuracy and fairness of predictive models compared to the COMPAS system. In this investigation, we will:

\begin{itemize}[]
	\item Analyse the COMPAS dataset and its predictions.
	\item Prepare the dataset for machine learning through cleaning, transformation, and feature engineering.
	\item Train and evaluate a neural network, a logistic regressor and use a k-nearest neighbour to obtain our recidivism risk.
	\item Investigate potential biases and ethical implications in predictions.
\end{itemize}